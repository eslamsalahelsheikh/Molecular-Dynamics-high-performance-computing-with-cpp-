\chapter{Implementation}\label{chap:Implementation}

\section{Velocity-Verlet integrator}
\subsection{Test strategy for Verlet integrator}
We test the Verlet integrator by comparing the results of the Verlet integrator with the results of the analytical solution of the equations of motion of a particle. We use the following equations of motion for a particle:
\begin{equation}
    \frac{d^2 x}{dt^2} = -\frac{k}{m}x
\end{equation}
\begin{equation}
    \frac{d^2 v}{dt^2} = -\frac{k}{m}v
\end{equation}
where $x$ is the position of the particle, $v$ is the velocity of the particle, $m$ is the mass of the particle, and $k$ is the spring constant. We use the following initial conditions:
\begin{equation}
    x(0) = 1
\end{equation}
\begin{equation}
    v(0) = 0
\end{equation}
\begin{equation}
    x'(0) = 0
\end{equation}
\begin{equation}
    v'(0) = 1
\end{equation}
where $x'$ is the velocity of the particle, and $v'$ is the acceleration of the particle. We use the following values for the mass of the particle and the spring constant: 
\begin{equation}
    m = 1
\end{equation}
\begin{equation}
    k = 1
\end{equation}
We use the following values for the time step:
\begin{equation}
    \Delta t = 0.01
\end{equation}
We use the following values for the number of steps:
\begin{equation}
    N = 1000
\end{equation}
We use the following values for the initial time:
\begin{equation}
    t_0 = 0
\end{equation}
We use the following values for the final time:
\begin{equation}
    t_f = 10
\end{equation}
We use the following values for the initial position:
\begin{equation}
    x_0 = 1
\end{equation}
We use the following values for the initial velocity:
\begin{equation}
    v_0 = 0
\end{equation}
We use the following values for the initial acceleration:

\section{Lennard-Jones potential force}
\subsection{Derivation of the analytical expression for the forces of the Lennard-Jones potential}

\section{Berendsen thermostat}
\subsection{Test strategy for Berendsen thermostat}

\section{Embedded atom method}
\section{units and specification of the time unit}
\subsection{time step for the gold potential}


\section{Neighbor List}

\section{Parallelization using MPI}

Explain the math and introduce notation.
\input{figures/Implementation/algBackpropagation}
\input{figures/Implementation/figTikz}